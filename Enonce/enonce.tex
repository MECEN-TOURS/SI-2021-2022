
\documentclass[11pt]{article}

\usepackage[frenchb]{babel}
\usepackage[utf8]{inputenc}
\usepackage[T1]{fontenc}
\usepackage{a4wide}

\usepackage{amsmath}
\usepackage{amsthm}
\usepackage{amstext}
\usepackage{amsfonts}
\usepackage{amssymb}
\usepackage{dsfont}
\usepackage{stmaryrd}

\usepackage{a4wide}
\usepackage{graphicx}
\usepackage{hyperref}
\usepackage[notref]{showkeys}

\usepackage{import}
\usepackage{xifthen}
\usepackage{pdfpages}
\usepackage{transparent}

\newcommand{\incfig}[1]{%
    \def\svgwidth{0.95 \columnwidth}
    \import{./figures/}{#1.pdf_tex}
}

\DeclareMathOperator*{\argmax}{arg\,max}

\newtheorem{exercice}{Exercice}
\newtheorem{question}{Question}

\title{Python pour le MECEN 2021}
\author{}

\begin{document}
\maketitle

\section{Du côté du consommateur}%
\label{sec:du_cote_du_consommateur}

\paragraph{Fonction d'utilité.} On se donne des paramètres réels $a>0$ et $b > |d| > 0$.
On introduit maintenant la fonction d'utilité $U: \mathbb{R}^3 \mapsto \mathbb{R}$ par
\begin{equation} \label{eq:1}
    \forall (q_0, q_1, q_2) \in \mathbb{R}^3,\quad U(q_0, q_1, q_2):= q_0 + a (q_1 + q_2) - \frac{bq_1^2+bq_2^2+2dq_1q_2}{2}.
\end{equation}
Ici $q_0, q_1, q_2$ sont les quantitées consommées de trois bien (de type 0, 1 et 2).

\begin{question}
    On pourra chercher à dessiner les ensembles de niveaux de $U$ avec des sliders pour ajuster les valeurs des paramètres.
\end{question}

\paragraph{Contrainte sur le revenu.} On considère $p_1$  et $p_2$  des réels positifs correspondant au prix unitaire des biens de type 1 et 2.
On se donne également un réel positif $R$ représentant le revenu global du consommateur.

\paragraph{Prise de décision rationnelle.} La répartition de consommation la plus avantageuse est alors la solution du programme suivant.
\begin{equation} \label{eq:2}
    \begin{cases}
        \argmax u(q_0, q_1, q_2),\\
        q_0, q_1, q_2 \geq 0,\\
        q_0 + p_1 q_1 + p_2q_2 \leq R.
    \end{cases}
\end{equation}

\begin{question}
    On pourra reprendre la visualisation précédente pour ajouter le tétraédre de contrainte et des sliders correspondant aux nouveaux paramètres.
\end{question}

\begin{question}
    Montrer que le programme précédent permet de définir génériquement $\tilde{q_0},\tilde{q_1}, \tilde{q_2}$ des fonctions des prix unitaires $p_1, p_2$.
\end{question}

\begin{question}
    Faites une visualisation des graphes des trois fonctions avec des sliders représentant les paramètres.
\end{question}

\begin{question}
    Déterminer des hypothèses sous lesquelles on peut transformer le système 
    \begin{equation} \label{eq:3}
        \begin{cases}
            \tilde{q_0}(p_1, p_2) = q_0,\\
            \tilde{q_1}(p_1, p_2) = q_1,\\
            \tilde{q_2}(p_1, p_2) = q_2,
        \end{cases}
    \end{equation}
    en 
    \begin{equation}
        \begin{cases}
            \tilde{p_1}(q_0, q_1, q_2)=p_1,\\
            \tilde{p_2}(q_0, q_1, q_2)=p_2.
        \end{cases}
    \end{equation}
\end{question}
\section{Concurrence en prix}%
\label{sec:concurrence_en_prix}

\paragraph{Description de l'économie.} Dans cette économie, il y a deux entreprises.
\begin{itemize}
    \item[$\bullet$] L'entreprise 1 produit des produits de type 1 au cout unitaire de revient $c_1>0$ et le vend au prix unitaire $p_1$.
    \item[$\bullet$] L'entreprise 2 produit des produits de type 2 au cout unitaire de revient $c_2>0$ et le vend au prix unitaire $p_2$.
\end{itemize}

\paragraph{Profits.}On a alors les niveaux de profits de chaque entreprise donnée par les fonctions
\begin{equation} \label{eq:6}
    \forall (p_1, p_2)\in \mathbb{R}_+^2,\quad\Pi_1(p_1, p_2):= (p_1 - c_1) \tilde{q_1}(p_1, p_2),
\end{equation}
et
\begin{equation} \label{eq:7}
    \forall (p_1, p_2)\in \mathbb{R}_+^2,\quad\Pi_2(p_1, p_2):= (p_2 - c_2) \tilde{q_2}(p_1, p_2).
\end{equation}
\paragraph{Décision rationnelle} Chaque entreprise cherchant à maximiser son profit cherche à résoudre le programme 
\begin{equation} \label{eq:4}
    \begin{cases}
        \argmax \Pi_1(p_1),\\
        p_1\geq 0.
    \end{cases}
\end{equation}
\begin{equation} \label{eq:5}
    \begin{cases}
        \argmax \Pi_2(p_2),\\
        p_2\geq 0.
    \end{cases}
\end{equation}

\begin{question}
    Montrer que les programmes précédents fournissent des fonctions de réactions $r_1$ au prix $p_2$ (resp. $r_2$ au prix $p_1$).
\end{question}

\begin{question}
    Montre qu'il existe un équilibre de Nash $(p^*_1, p^*_2)$  solution de 
    \begin{equation} \label{eq:8}
        \begin{cases}
            r_1(p^*_2)=p^*_1,\\
            r_2(p^*_1)=p^*_2.
        \end{cases}
    \end{equation}
\end{question}

\begin{question}
    Visualiser les courbes de réactions (et donc les équilibres de Nash qui sont les intersections) avec des sliders permettant de déterminer les paramètres.
\end{question}
\section{Concurrence en quantité}%
\label{sec:concurrence_en_quantite}

Les variables de décision des entreprises sont maintenant les quantités produites $q_1, q_2$.
On obtient donc des problèmes
\begin{equation} \label{eq:9}
    \begin{cases}
        \argmax \tilde{p_1}(q_0, q_1, q_2) q_1 - c_1 q_1,\\
        q_1 \geq 0
    \end{cases}
\end{equation}
et 
\begin{equation} \label{eq:10}
    \begin{cases}
        \argmax \tilde{p_2}(q_0, q_1, q_2) q_2 - c_2 q_2,\\
        q_2 \geq 0
    \end{cases}
\end{equation}
\begin{question}
Montrer que les programmes fournissent des fonctions de réactions $s_1$ à la quantité produite $q_2$ (resp. $s_2$ à $q_1$).
\end{question}

\begin{question}
    Montre qu'il existe un équilibre de Nash $(q_1^*, q_2^*)$ solution de 
    \begin{equation} \label{eq:11}
        \begin{cases}
            s_1(q_2^*)=q_1^*,\\
            s_2(q_1^*)=q_2^*.
        \end{cases}
    \end{equation}
\end{question}

\begin{question}
    Visualiser les courbes de réactions (et donc les équilibres de Nash qui sont les intersections) avec des sliders permettant de déterminer les paramètres.
\end{question}
\end{document}
